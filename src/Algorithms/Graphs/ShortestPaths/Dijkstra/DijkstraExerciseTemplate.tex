\textbf{\LARGE{\color{tumgadPurple}Dijkstra's Algorithm}}\\
\\
\noindent
Given the following graph:

\begin{center}
    \begin{tikzpicture}[
        > = stealth, % arrow head style
        auto,
        node distance = 3cm, % distance between nodes
        semithick % line style
    ]
        \node[state, fill=tumgadBlue] (00) {%00
        };
        \node[state, draw=none] (01) [right of=00] {%01
        };
        \node[state, draw=none] (02) [right of=01] {%02
        };
        \node[state, draw=none] (03) [right of=02] {%03
        };
        \node[state, draw=none] (04) [right of=03] {%04
        };
        \node[state, draw=none] (05) [right of=04] {%05
        };
        \node[state, draw=none] (06) [right of=05] {%06
        };

        \node[state, draw=none] (10) [below of=00] {%10
        };
        \node[state, draw=none] (11) [right of=10] {%11
        };
        \node[state, draw=none] (12) [right of=11] {%12
        };
        \node[state, draw=none] (13) [right of=12] {%13
        };
        \node[state, draw=none] (14) [right of=13] {%14
        };
        \node[state, draw=none] (15) [right of=14] {%15
        };
        \node[state, draw=none] (16) [right of=15] {%16
        };

        \node[state, draw=none] (20) [below of=10] {%20
        };
        \node[state, draw=none] (21) [right of=20] {%21
        };
        \node[state, draw=none] (22) [right of=21] {%22
        };
        \node[state, draw=none] (23) [right of=22] {%23
        };
        \node[state, draw=none] (24) [right of=23] {%24
        };
        \node[state, draw=none] (25) [right of=24] {%25
        };
        \node[state, draw=none] (26) [right of=25] {%26
        };

        \node[state, draw=none] (30) [below of=20] {%30
        };
        \node[state, draw=none] (31) [right of=30] {%31
        };
        \node[state, draw=none] (32) [right of=31] {%32
        };
        \node[state, draw=none] (33) [right of=32] {%33
        };
        \node[state, draw=none] (34) [right of=33] {%34
        };
        \node[state, draw=none] (35) [right of=34] {%35
        };
        \node[state, draw=none] (36) [right of=35] {%36
        };

        \node[state, draw=none] (40) [below of=30] {%40
        };
        \node[state, draw=none] (41) [right of=40] {%41
        };
        \node[state, draw=none] (42) [right of=41] {%42
        };
        \node[state, draw=none] (43) [right of=42] {%43
        };
        \node[state, draw=none] (44) [right of=43] {%44
        };
        \node[state, fill=tumgadBlue] (45) [right of=44] {%45
        };

        %$CONNECTIONS$
    \end{tikzpicture}
\end{center}
a) Follow the \href{https://ossner.github.io/TUMGAD/src/Algorithms/Graphs/ShortestPaths/Dijkstra/Dijkstra}{\underline{Algorithm of Dijkstra}} as covered in the lecture and find the shortest path from node $A$ to node $MAXCHAR$.\\
Note the algorithms priority queue after every step, as well as the changes made to the queue in the step.\\
If nodes have equal priorities, the algorithm follows the node that was inserted first.\\
\newpage
\begin{center}
    \begin{tabular}{|P{8cm}|P{6cm}|}
        \hline
        Priority Queue & Updates in Queue\\
        \hline
        \hline
        &\\[1.5ex]
        \hline
        &\\[1.5ex]
        \hline
        &\\[1.5ex]
        \hline
        &\\[1.5ex]
        \hline
        &\\[1.5ex]
        \hline
        &\\[1.5ex]
        \hline
        &\\[1.5ex]
        \hline
        &\\[1.5ex]
        \hline
        &\\[1.5ex]
        \hline
        &\\[1.5ex]
        \hline
        &\\[1.5ex]
        \hline
        &\\[1.5ex]
        \hline
        &\\[1.5ex]
        \hline
        &\\[1.5ex]
        \hline
        &\\[1.5ex]
        \hline
    \end{tabular}
\end{center}
b) What is the shortest path between the nodes and how long is it?\\
\begin{center}
    \noindent\fbox{
        \parbox{14cm}{
            \vspace{0.15cm}
            A $\rightarrow$
            \hspace{12cm}
            \vspace{0.2cm}
            \\
            Length: \underline{\hspace{1cm}}
            \vspace{0.15cm}
        }
    }
\end{center}
